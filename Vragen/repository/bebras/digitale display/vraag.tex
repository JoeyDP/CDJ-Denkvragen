\documentclass[12pt, a4paper]{article}

\usepackage{import}
\usepackage{standalone}

\usepackage[top=4.5cm, right=2cm, bottom=2.7cm, left=2cm]{geometry}

\usepackage{wrapfig}
\usepackage{tabulary}
\usepackage{float}
\usepackage{pifont}
\usepackage{background}
\usepackage{tikz}
\usepackage{mathtools}
\usepackage{multirow}
\usepackage{listings}


\pagestyle{empty}
\setlength{\parindent}{0pt}

\begin{document}
	\newcommand{\digit}[1]{\raisebox{-0.4\height}{\includegraphics[width=0.5cm]{#1}}}
	\newcommand{\Digit}[1]{\raisebox{-0.4\height}{\includegraphics[width=1cm]{#1}}}
	
	\begin{minipage}{\textwidth}
		\section{Digitale Display \hfill\small Bron: Bebras}
			
			Een digitale display bestaat uit 7 lichtsegmenten die elk afzonderlijk kunnen ingeschakeld zijn of uitgeschakeld. Op die manier kunnen we alle cijfers van 0 t.e.m. 9 voorstellen, zoals hieronder ge\"illusteerd. 
			\begin{figure}[H]
				\centering
				\Digit{0}
				\Digit{1}
				\Digit{2}
				\Digit{3}
				\Digit{4}
				\Digit{5}
				\Digit{6}
				\Digit{7}
				\Digit{8}
				\Digit{9}
			\end{figure}
			We laten twee bewerkingen toe op een dergelijke display.

			\begin{tabulary}{\linewidth}{|L|L|}
				\hline
				De bewerking ''Inverteer'' & De bewerking ''Combineer'' \\ \hline
				Schakelt de segmenten in die waren uitgeschakeld en schakelt de segmenten uit die waren ingeschakeld.
				Bijvoorbeeld:
				
				Inverteer(\Digit{3}) geeft \Digit{1}
				 &
				Werkt in op twee displays en produceert een nieuw display waarop alle segmenten worden ingeschakeld die op minstens \'e\'en van de displays zijn ingeschakeld, en enkel deze. 
				
				Bijvoorbeeld: 
				
				Combineer(\Digit{4}, \Digit{5}) geeft \Digit{9}
				\\ \hline
			\end{tabulary} \\
			
			Welke van de onderstaande bewerkingen heeft niet \Digit{8} als resultaat?
			
			\begin{table}[H]
				\centering
				\begin{tabulary}{\linewidth}{|C|C|}
					\hline
					\textbf{A}
					
					Combineer(Inverteer( \digit{7} ), \digit{2} ) \vspace{0.1cm}
					&
					\textbf{B} 
					
					Combineer( \digit{5} , \digit{2} ) \vspace{0.1cm}
					\\ \hline
					\textbf{C}
					
					Combineer( Inverteer( \digit{0} ), \digit{0} ) \vspace{0.1cm}
					&
					\textbf{D}
					
					Combineer( Inverteer( \digit{5} ), \digit{8} ) \vspace{0.1cm}
					\\ \hline 
				\end{tabulary}
			\end{table}
	\end{minipage} \\ \\
	
\end{document}	