\documentclass[12pt, a4paper]{article}

\usepackage{import}
\usepackage{standalone}

\usepackage[top=4.5cm, right=2cm, bottom=2.7cm, left=2cm]{geometry}

\usepackage{wrapfig}
\usepackage{tabulary}
\usepackage{float}
\usepackage{pifont}
\usepackage{background}
\usepackage{tikz}
\usepackage{mathtools}
\usepackage{multirow}
\usepackage{listings}


\pagestyle{empty}
\setlength{\parindent}{0pt}

\begin{document}
	\begin{minipage}{\textwidth}
		\section{Anonimisatie \hfill\small Bron: Bebras}
			Medische dossiers bevatten vaak gevoelige gegevens die echt niet publiek mogen gemaakt worden. Nochtans kan het vaak interessant zijn om statistieken te bestuderen van een grote groep pati\"enten, bijvoorbeeld om na te gaan hoeveel bepaalde ziektes voorkomen en hoe ze zich verspreiden. Om die redenen kan het gebeuren dat een ziekenhuis tcoh informatie publiek maakt over zijn pati\"enten, maar dan ''geanonimiseerd''. Tabel 1 hieronder geeft bijvoorbeeld de gegevens weer van alle pati\"enten die op een $1^{ste}$ januari zijn geboren. \\
			
			Het is nu niet zo moeilijk om los hiervan, bijvoorbeeld via het gemeentehuis, een lijst op de kop te tikken van alle inwoners van een bepaalde gemeente. Tabel 2 hieronder geeft de lijst van alle inwoners van de gemeente met postnummer 9834 die geboren zijn op een $1^{ste}$ januari.
			
			\begin{minipage}{0.5\linewidth}
				\begin{table}[H]
					\begin{tabulary}{\linewidth}{|L|L|L|L|}
						\hline
						\textbf{Geboorte} & \textbf{G} & \textbf{Post} & \textbf{Ziekte} \\ \hline
						01/01/1974 & M & 9400 & Suikerziekte \\ \hline
						01/01/1976 & M & 9834 & Nierstenen \\ \hline
						01/01/1976 & V & 9400 & Borstkanker \\ \hline
						01/01/1976 & V & 9400 & Hepatitis \\ \hline
						01/01/1984 & V & 9834 & Hartproblemen \\ \hline
						01/01/1985 & V & 8300 & Niersteken \\ \hline
						01/01/1987 & V & 2340 & Huidkanker \\ \hline
						01/01/1998 & M & 9834 & Suikerziekte \\ \hline
						01/01/1998 & V & 9834 & Longontsteking \\ \hline
					\end{tabulary}
					\centering Tabel 1: Pati\"enten geboren op een $1^{ste}$ januari
				\end{table}
			\end{minipage} \hfill
			\begin{minipage}{0.48\linewidth}
				\begin{table}[H]
					\begin{tabulary}{\linewidth}{|L|L|L|L|}
						\hline
						\textbf{Post} & \textbf{Geboorte} & \textbf{G} & \textbf{Naam} \\ \hline
						9834 & 01/01/1958 & V & Ann Coulier \\ \hline
						9834 & 01/01/1976 & M & Bart Saelens \\ \hline
						9834 & 01/01/1976 & M & Geert Peeters \\ \hline
						9834 & 01/01/1984 & V & Diane Gerbers \\ \hline
						9834 & 01/01/1984 & V & Kaat Perron \\ \hline
						9834 & 01/01/1998 & V & Melanie Bertels \\ \hline
						9834 & 01/01/1998 & M & Rik Devriendt \\ \hline
						9834 & 01/01/1998 & V & Iris Isebaert \\ \hline
						9834 & 01/01/1999 & M & Mark De Korte \\ \hline
					\end{tabulary}
					\centering Tabel 2: Inwoners geboren op een $1^{ste}$ januari
				\end{table}
			\end{minipage} \\
			
			Door de informatie te combinerem van beide tabellen hierboven kan met een persoon vinden waarvan men kan zeker zijn dat hij of zij ziek is. Wat is de naam van deze persoon?

			\begin{table}[H]
				\centering 
				\begin{tabulary}{\linewidth}{|C|C|C|C|C|}
					\hline
					\textbf{A} \vspace{0.1cm}
					
					Rik Devriendt
					&
					\textbf{B} \vspace{0.1cm}
					
					Ann Coulier	
					&
					\textbf{C} \vspace{0.1cm}
					
					Bart Saelens	
					&
					\textbf{D} \vspace{0.1cm}
					
					Mark De Korte	
					&
					\textbf{E} \vspace{0.1cm}
					
					Geert Peeters
					\\ \hline
					\textbf{F} \vspace{0.1cm}
					
					Kaat Perron	
					&
					\textbf{G} \vspace{0.1cm}
					
					Melanie Bertels	
					&
					\textbf{H} \vspace{0.1cm}
					
					Luc Vandiest	
					&
					\textbf{I} \vspace{0.1cm}
					
					Diane Gerbers	
					&
					\textbf{J} \vspace{0.1cm}
					
					Iris Isebaert
					\\ \hline
				\end{tabulary}
			\end{table}

	\end{minipage}
		
\end{document}