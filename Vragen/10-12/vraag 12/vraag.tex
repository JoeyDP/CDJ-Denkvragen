\documentclass[12pt, a4paper]{article}

\usepackage{import}
\usepackage{standalone}

\usepackage[top=4.5cm, right=2cm, bottom=2.7cm, left=2cm]{geometry}

\usepackage{wrapfig}
\usepackage{tabulary}
\usepackage{float}
\usepackage{pifont}
\usepackage{background}
\usepackage{tikz}
\usepackage{mathtools}
\usepackage{multirow}
\usepackage{listings}


\pagestyle{empty}
\setlength{\parindent}{0pt}

\begin{document}
	\begin{minipage}{\textwidth}
		\section{Bebrocarina \hfill\small Bron: Bebras}
			
			De bebrocarina is een muziekinstrument waarop men, \textbf{nadat men \'e\'en noot heeft gespeeld, daarna enkel nog dezelfde noot kan spelen, of een noot \'e\'en toon lager of een noot \'e\'en toon hoger}. Een partituur voor een bebrocarina wordt daarom geschreven met slechts drie verschillende symbolen:
			
			\begin{table}[H]
				\setlength\parindent{24pt}
				\begin{tabular}{l}
					$=$ betekent '' speel dezelfde noot als de vorige '', \\
					$+$ betekent '' speel de noot \'e\'en toon hoger '', \\
					$-$ betekent '' speel de noot \'e\'en toon lager ''.
				\end{tabular}
			\end{table}
			
			De partituur geeft niet aan met welke noot je moet beginnen, ze toont enkel hoe de noten elkaar opvolgen.
			
			Bijvoorbeeld, de partituur ''$- +$'' vertelt je dat je eerst een willekeurige noot mag spelen, daarna een noot die \'e\'en toon lager is dan de eerste noot, en tenslotte een noot die opnieuw \'e\'en noot hoger is dan de tweede (en dus dezelfde noot als de eerste).
			
			Als je weet dat een bebrocarina maar \textbf{zes verschillende noten} kan spelen, welke van de volgende partituren kan dan niet op een bebrocarina gespeeld worden?
			
			\begin{table}[H]
				\centering
				\begin{tabular}{|c l|}
					\hline
					\textbf{A} & $- - - = + - = - - = = = +$ \\ 
					\textbf{B} & $- - - - - = + + + + + = - - - - -$ \\  
					\textbf{C} & $- - + - - + - - = - + - -$ \\ 
					\textbf{D} & $+ = = = + = = = + = = = + = = = +$ \\ 
					\hline
				\end{tabular}
			\end{table}
	\end{minipage} \\ \\
	
\end{document}	